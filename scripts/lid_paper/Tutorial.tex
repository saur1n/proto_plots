\documentclass[]{article}
\usepackage{lmodern}
\usepackage{amssymb,amsmath}
\usepackage{ifxetex,ifluatex}
\usepackage{fixltx2e} % provides \textsubscript
\ifnum 0\ifxetex 1\fi\ifluatex 1\fi=0 % if pdftex
  \usepackage[T1]{fontenc}
  \usepackage[utf8]{inputenc}
\else % if luatex or xelatex
  \ifxetex
    \usepackage{mathspec}
  \else
    \usepackage{fontspec}
  \fi
  \defaultfontfeatures{Ligatures=TeX,Scale=MatchLowercase}
\fi
% use upquote if available, for straight quotes in verbatim environments
\IfFileExists{upquote.sty}{\usepackage{upquote}}{}
% use microtype if available
\IfFileExists{microtype.sty}{%
\usepackage[]{microtype}
\UseMicrotypeSet[protrusion]{basicmath} % disable protrusion for tt fonts
}{}
\PassOptionsToPackage{hyphens}{url} % url is loaded by hyperref
\usepackage[unicode=true]{hyperref}
\hypersetup{
            pdftitle={LI Detector Tutorial},
            pdfauthor={Saurin Parikh},
            pdfborder={0 0 0},
            breaklinks=true}
\urlstyle{same}  % don't use monospace font for urls
\usepackage[margin=1in]{geometry}
\usepackage{graphicx,grffile}
\makeatletter
\def\maxwidth{\ifdim\Gin@nat@width>\linewidth\linewidth\else\Gin@nat@width\fi}
\def\maxheight{\ifdim\Gin@nat@height>\textheight\textheight\else\Gin@nat@height\fi}
\makeatother
% Scale images if necessary, so that they will not overflow the page
% margins by default, and it is still possible to overwrite the defaults
% using explicit options in \includegraphics[width, height, ...]{}
\setkeys{Gin}{width=\maxwidth,height=\maxheight,keepaspectratio}
\IfFileExists{parskip.sty}{%
\usepackage{parskip}
}{% else
\setlength{\parindent}{0pt}
\setlength{\parskip}{6pt plus 2pt minus 1pt}
}
\setlength{\emergencystretch}{3em}  % prevent overfull lines
\providecommand{\tightlist}{%
  \setlength{\itemsep}{0pt}\setlength{\parskip}{0pt}}
\setcounter{secnumdepth}{0}
% Redefines (sub)paragraphs to behave more like sections
\ifx\paragraph\undefined\else
\let\oldparagraph\paragraph
\renewcommand{\paragraph}[1]{\oldparagraph{#1}\mbox{}}
\fi
\ifx\subparagraph\undefined\else
\let\oldsubparagraph\subparagraph
\renewcommand{\subparagraph}[1]{\oldsubparagraph{#1}\mbox{}}
\fi

% set default figure placement to htbp
\makeatletter
\def\fps@figure{htbp}
\makeatother


\title{LI Detector Tutorial}
\author{Saurin Parikh}
\date{08/15/2020}

\begin{document}
\maketitle

{
\setcounter{tocdepth}{2}
\tableofcontents
}
\section{STEP 1}\label{step-1}

\subsection{Photos, Plate maps and strain-id to orf-name
table}\label{photos-plate-maps-and-strain-id-to-orf-name-table}

\begin{enumerate}
\def\labelenumi{\arabic{enumi}.}
\tightlist
\item
  Photos from the experiment need to be organized in a heirarchy of
  folders

  \begin{itemize}
  \tightlist
  \item
    Example

    \begin{itemize}
    \tightlist
    \item
      if you were conducting an experiment using the mutant collection
    \item
      the experiment had two parallel arms going from 384 density plates
      to 1536 density plates to 6144 density plates
    \item
      photos for the 384 and 1536 density plates were taken at
      saturation and those for 6144 density plates were taken at 0, 4
      and 12 hours
    \item
      then the folder heirarchy would be as follows:

      \begin{itemize}
      \tightlist
      \item
        Experiment

        \begin{itemize}
        \tightlist
        \item
          Arm \#1

          \begin{itemize}
          \tightlist
          \item
            384 density

            \begin{itemize}
            \tightlist
            \item
              36h
            \end{itemize}
          \item
            1536 density

            \begin{itemize}
            \tightlist
            \item
              20h
            \end{itemize}
          \item
            6144 density

            \begin{itemize}
            \tightlist
            \item
              00h
            \item
              04h
            \item
              12h
            \end{itemize}
          \end{itemize}
        \item
          Arm \#2

          \begin{itemize}
          \tightlist
          \item
            384 density

            \begin{itemize}
            \tightlist
            \item
              36h
            \end{itemize}
          \item
            1536 density

            \begin{itemize}
            \tightlist
            \item
              20h
            \end{itemize}
          \item
            6144 density

            \begin{itemize}
            \tightlist
            \item
              00h
            \item
              04h
            \item
              12h
            \end{itemize}
          \end{itemize}
        \end{itemize}
      \end{itemize}
    \end{itemize}
  \end{itemize}
\end{enumerate}

\end{document}
